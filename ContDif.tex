\section{Continuidad y Diferenciabilidad}

Una función de variable real y valor real, es un subconjunto de
$\R\times\R$, por lo que, es una entidad de $\Rh$. De igual forma, una
función $f$ se extiende a $\st{f}$ la cual, por el teorema \ref{theo:FT}
y el lema \ref{lema:stR} es un subconjunto de $\st{\R}\times\st{\R}$.
De la misma forma, si $A=\dom f$, para todo $x\in A$, $f(a) = \st{f(a)}$.
Cabe resaltar, que las propiedades de $f$ se mantienen en $\st{\R}$
mientras estas se puedan expresar en ambos lenguajes.

La definición de $\lim_{x\to c} f(x) = a$ en el análisis estándar es
\[
  \Forall{\e}[\e>0]{
    \Exists{\delta}[\delta>0]{
      \Forall{x}[x\in\R\land 0< |x-c|<\delta]{|f(x)-a|<\e}
    }
  }  
\]

De la misma forma que se hizo con las sucesiones, se obtiene una
definición análoga para el análisis no estándar:
\[\Forall{x}[x\in\mu(c)-\{c\}]{\st{f(x)} =_1 a}\]
Definiendo la suma de conjuntos como:
$A+B = \set{a+b}{a\in A \land b\in B}$, se tiene que, para todo
$r\in\R$, $\mu(r) = M_1 + \{r\}$. Usando esto, se puede
reescribir la definición de límite de la siguiente forma
\[\Forall{x}[x\in M_1-\{0\}]{\st{f(c+x)}=_1 a}\]

Nótese que para la continuidad de $f$ en $c$, se tiene respectivamente
la siguiente definición:
\[\Forall{x}[x\in M_1]{\st{f(c+x)}=_1 f(c)}\]

La derivada, en su definición, tiene dos ``versiones'', las cuales son
completamente equivalentes. La derivada de $f$ en $x$ existe si y solo
si los siguientes límites existen (basta con uno, pues son el mismo límite):

\begin{enumerate}
  \item $\lim{h\to 0}\frac{f(x + h) - f(x)}{h}$
  \item $\lim{t\to x}\frac{f(t) - f(x)}{t - x}$
\end{enumerate}

Por simplicidad se considerará el resultado de estos límites como $L$.
En el análisis no estándar, análogamente, la derivada de $f$ en $x$
existe si y solo si, para todo $h\in M_1-\{0\}$:
\[\frac{\st{f(x + h)} - f(x)}{h} =_1 L\]

De igual forma, se puede definir $f'(x)$ como la parte estándar del
cociente que define el límite:
\[f'(x) = \sd{\frac{\st{f(x + h)} - f(x)}{h}}\]

Como ejemplo, se mostrará el uso de estas definiciones para demostrar que
la diferenciabilidad en un punto implica continuidad en ese punto.
Sean $x\in\R$, $f$ una función derivable en $x$ y $h\in M_1-\{0\}$.
\begin{longderivation}<.8>
    {\st{f(x+h)} - f(x)}\\
  =_1\\
    {\frac{\st{f(x + h)} - f(x)}{h}\,h}\\
  =_1\\
    {f'(x)\,h}\\
  =_1\\
    {0}
\end{longderivation}

Nótese que la demostración se puede replicar en el análisis estándar,
ya que la relación `$=_1$' es análoga a decir que algo tiende a un
resultado bajo cierta condición, esta condición vienen a ser los
hiperreales que están operando.
